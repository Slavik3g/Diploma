\sectionCenteredToc{ВВЕДЕНИЕ}
\label{sec:intro}

Новейшие достижения в области нейросетей привели к революции в области искусственного интеллекта. Глубокие нейросети, изучающие методы машинного обучения и обработки данных, используются во многих приложениях во всем мире. Они находят применение в широком спектре задач, включая, но не ограничиваясь, распознаванием образов, классификацией данных и созданием нового контента. Благодаря этому нейронные сети стали неотъемлемой частью разработки современных приложений, зарекомендовав себя во множестве направлений, начиная от автономного вождения и заканчивая онкологической диагностикой и разработкой игр. Это непрерывно растущее влияние подчеркивает их значимость в текущем и будущем развитии технологий.

Одной из ключевых особенностей глубоких нейросетей является их способность извлекать сложные признаки из необработанных данных без необходимости ручного определения признаков или экспертного вмешательства. Это значительно упрощает и ускоряет процессы анализа данных, делая их доступными для более широкого круга исследований и практических приложений.

В области компьютерного зрения значительные успехи были достигнуты благодаря применению сверточных нейросетей и усовершенствованным методам глубокого обучения. Эти технологии играют ключевую роль в улучшении способов обработки и анализа изображений. Сверточные нейросети, в частности, эффективно способствуют автоматизации таких процессов, как распознавание и классификация различных объектов на изображениях. Они позволяют системам не только точно идентифицировать и классифицировать объекты в широком ряду условий, но также обрабатывать изображения на уровне, который ранее был недоступен.

Кроме того, нейросети поддерживают выполнение более сложных и технически требовательных задач, таких как семантическая сегментация. Семантическая сегментация включает в себя процесс разделения изображения на части, которые соответствуют различным категориям объектов, что позволяет системам глубже понимать контекст и содержание визуальных данных.

Эти успехи в обработке изображений способствовали созданию моделей, которые могут не только анализировать визуальные данные на углубленном уровне, но и создавать новые или модифицировать уже существующие изображения, тем самым расширяя возможности в области креативности и визуального дизайна. Это облегчает работу многих специалистов в различных сферах, включая графический дизайн, архитектуру, производство видео и маркетинг, предоставляя им мощные инструменты для реализации творческих идей и стратегических концепций.

Данные нейронные сети облегчают работу многих специалистов в сфере графического дизайна, архитектуры, видеопроизводства и даже в маркетинге, предоставляя мощные инструменты для воплощения творческих идей и стратегических задумок в реальность. Особенно это касается разработки уникальных визуальных концепций, которые раньше требовали значительных временных затрат и усилий. С помощью алгоритмов глубокого обучения специалисты могут экспериментировать с дизайном, тонко настраивать элементы визуализации и даже создавать реалистичные 3D-модели, что ранее казалось недостижимым без глубоких знаний в области программирования и компьютерной графики.

В архитектуре нейросети помогают автоматизировать создание моделей и оптимизировать дизайн, ускоряя процессы разработки и способствуя внедрению инноваций. В производстве видео технологии позволяют улучшать качество видео, добавлять сложные визуальные эффекты, повышая его привлекательность. В маркетинге и рекламе алгоритмы способствуют более точной персонализации и целенаправленности кампаний, укрепляя связь с аудиторией.

Исходя из всего вышеперечисленного, становится понятным, что редактирование изображений с помощью нейронных сетей является задачей первостепенной важности. Они открывают огромный горизонт возможностей для разных сфер деятельности.

Целью данного дипломного проекта является исследование и разработка передовых методик создания и модификации изображений при помощи последних достижений в области нейронных сетей. Проект направлен на улучшение взаимодействия пользователей с визуальным контентом, предлагая им интуитивно понятные инструменты для создания новаторских и выразительных визуальных произведений.

В соответствии с поставленной целью были определены следующие задачи:
\begin{enumerate_num}
    \item Исследование и анализ современных нейронных сетей.
    \item Обзор аналогов и постановка задачи.
    \item Проектирование и реализация архитектуры программного модуля.
    \item Тестирование работоспособности модуля.
    \item Расчет экономических показателей дипломного проекта.
    \item Написание руководства пользователя.
\end{enumerate_num}
