\sectionCenteredToc{ЗАКЛЮЧЕНИЕ}
\label{sec:outro}

Во время выполнения дипломного проектирования была изучена область генерации и изменения изображений по текстовым описаниям. Определена цель разработки, задачи, сформирован перечень 
функциональных требований, предъявляемых к системе. Также были рассмотрены аналоги и на основе этих данных сформированы конкурентные преимущества разрабатываемой системы.

Был произведен анализ, выбраны средства реализации. Кроме того, была произведена декомпозиция приложения, в результате, разработана структурная схема, описывающая блоки приложения. Системный подход к разработке позволяет детально продумать архитектуру приложения и избежать проблем на этапе реализации. 

Выполнено технико-экономическое обоснование разработки и реализации программного продукта.

Реализованный программный модуль состоит из двух взаимодействующих нейросетей: одна отвечает за сегментацию изображений, выделяя отдельные элементы, а другая~-- за редактирование и создание новых элементов в изображении с помощью диффузных моделей, исходя из предоставленных масок. Данный модуль будет полезен в различных областях, включая искусство, дизайн и рекламу.

По итогам разработки была получена конечная версия продукта.

Модуль сопровождается пояснительной запиской, схемой программы, блок-схемами алгоритмов, диаграммой последовательности и структурной схемой.

% \citeMy{diplomnoe_proect, econom}