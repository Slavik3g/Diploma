% For page mirroring
\newgeometry{left=3cm,right=1.5cm,top=2.0cm,bottom=2.0cm,twoside}
    \begin{center}
      Министерство образования Республики Беларусь\\[1em]
      Учреждение образования\\
      БЕЛОРУССКИЙ ГОСУДАРСТВЕННЫЙ УНИВЕРСИТЕТ \\
      ИНФОРМАТИКИ И РАДИОЭЛЕКТРОНИКИ\\[1em]
    \end{center}

    \begin{center}
      Кафедра электронных вычислительных машин
    \end{center}

    \begin{flushright}
      \begin{minipage}{0.4\textwidth}
        \MakeUppercase{Утверждаю}\\
        Заведующий кафедрой ЭВМ\\
        \underline{\hspace*{2.2cm}} \headOfDepartmentShort \\
        <<\underline{\hspace*{1cm}}>> \underline{\hspace*{2.5cm}} \targetYear\ г.
      \end{minipage}\\[1em]
    \end{flushright}

    \begin{center}
      \textbf{ЗАДАНИЕ}\\
      \textbf{на дипломный проект}\\[1em]
    \end{center}

    \noindent
    Обучающемуся: \studentFullParental \\
    Курс: 4 \hspace*{2ex} Учебная группа: 050504 \\
    Специальность: 40 02 01 <<Вычислительные машины, системы и сети>>

    \vspace{1em}
    \noindent
    Тема дипломного проекта: \taskNameFull

    \vspace{1em}
    \noindent
    Утверждена ректором БГУИР \uniDecreeDate \ приказ \textnumero \ \uniDecreeNumber.

    \vspace{1em}
    \noindent
    Исходные данные к дипломному проекту:
    \begin{itemize}[label={}, itemindent=\parindent]
      \item 1 Среда разработки: Google Colaboratory.
      \item 2 Операционная система: Linux.
      \item 3 Языки программирования: Python, JavaScript.
      \item 4 Формат входных и выходных данных для фотографий: png, jpeg.
    \end{itemize}

    \vspace{1em}
    \noindent
    Перечень подлежащих разработке вопросов или краткое содержание расчетно-пояснительной записки:

    \indent Введение. 1. Обзор литературы. 2. Системное проектирование.
        3. Функциональное проектирование. 4. Разработка программных модулей.
        5. Программа и методика испытаний. 6. Руководство пользователя.
        7. \economicalPartName. Заключение. Список использованных источников. Приложения.

    \vspace{1em}
    \noindent
    Перечень графического материала (с точным указанием обязательных чертежей и графиков):

    \indent 1. Вводный плакат. Плакат. 2. \taskNameFull. Схема структурная. 3. \taskNameFull. Схема программы. 4. \taskNameFull. Диаграмма последовательности. 5. Функция автоматической сегментации. Схема программы. 6. Функция определения связей между масками. Схема программы. 7. Функция отображения масок на изображении. Схема программы. 8. Заключительный плакат. Плакат.

    \vspace{1em}
    \noindent
    Консультанты по дипломному проектированию (с указанием разделов, по которым они консультируют):\\
    \begin{tabular}{ @{}b{0.45\textwidth}b{0.30\textwidth}b{0.25\textwidth} }
      Консультант по дипломному проектированию: & \underline{\hspace*{4.75cm}} & \practiceDepartmentTutorShort
    \end{tabular}

    \vspace{1em}
    \noindent
    Содержание задания по экономической части: <<\economicalPartName>>.\\
    \begin{tabular}{ @{}b{0.45\textwidth}b{0.30\textwidth}b{0.25\textwidth} }
      Консультант по технико-экономическому обоснованию: & \underline{\hspace*{4.75cm}} & \diplomaEconomyTutorShort
    \end{tabular}

    \vspace{1em}
    \noindent
    Примерный календарный график выполнения дипломного проекта:\\
    \begin{tabular}
                {| >{\raggedright}m{0.448\textwidth}
                | >{\centering}m{0.10\textwidth}
                | >{\centering}m{0.17\textwidth}
                | >{\centering\arraybackslash}m{0.17\textwidth}|}
      \hline
        \centering Наименование этапов дипломного проекта
      & Объем этапа, $ \% $ & Срок выполнения этапа & Примечания \\
      \hline
      Подбор и изучение литературы. Сравнение аналогов. Уточнение задания на ДП & 10 & 25.03 – 30.03 & \\ \hline
      Структурное проектирование & 15 & 30.03 – 08.04 & \\ \hline
      Функциональное проектирование & 25 & 08.04 – 24.04 & \\ \hline
      Принципиальное проектирование & 20 & 24.04 – 08.05 & \\ \hline
      Программа и методика испытаний & 10 & 08.05 – 15.05 & \\ \hline
      Расчет экономической эффективности & 5 & 15.05 – 20.05 & \\ \hline
      Оформление пояснительной записки & 10 & 20.05 – 30.05 & \\ \hline
    \end{tabular}


    \vspace{1em}
    \noindent
    Дата выдачи задания: \taskStartDate\\
    Срок сдачи законченного дипломного проекта: \taskFinishDate

    \vspace{1em}
    \noindent
    % @{} for left margin removing in tabular
    \begin{tabular}{ @{}p{0.45\textwidth}p{0.30\textwidth}p{0.25\textwidth} }
      Руководитель дипломного проекта & \underline{\hspace*{4.75cm}} & \diplomaTutorShort \\
      && \\
      Подпись обучающегося & \underline{\hspace*{4.75cm}} &
    \end{tabular}


    \noindent
    Дата: \taskStartDate

  \newpage

\restoregeometry

